\section{Math in Computer Science}
\label{s:tosiek}

\subsection{Where do we use math in Computer Science?}
Actually I don't know. I'm only on the first year of studies. We all know that without math there wouldn't be computer science, because Post machine is just a mathematical model. Also PC architecture is just math (mostly logic). Moreover numerical methods are (almost) pure math. Machine learning and deep learning are in 90\% \textbf{matrices multiplication}. Computer graphic is just analytical geometry and algebra. So if you want to be the best IT specialist, you must know math. (For more see Table \ref{table:math}) \\ Libraries from programming languages which use math:
\begin{enumerate}
    \item In C++:
        \begin{itemize}
            \item cmath
            \item GLM
            \item math.h
            \item SFML
        \end{itemize}
    \item In python:
        \begin{itemize}
            \item numpy
            \item math
            \item SFML
            \item scipy
        \end{itemize}
\end{enumerate}
\begin{table}[h]
\centering
\begin{tabular}{|l|l|}
\hline
\textbf{Department of Computer Science} & \textbf{Used math}           \\ \hline
Computer Graphics               & Calculus             \\ \hline
Cybersecurity                  & Discrete mathematics \\ \hline
Artificial  Inteligence         & Linear algebra       \\ \hline
\end{tabular}
\caption{Usage of math in IT}
\label{table:math}
\end{table}
\subsection{Why we use math in Computer Science}
\underline{Yes} \\ 
And also, to make things \emph{easier}. Math is used for computing complexity of algorithms. You can also optimize algorithms if you can math. We also use math to approximate solutions for equations, which don't have accurate solutions. A simplest example of such equation is: \\
\begin{math}
e^x=x^2+1 
\end{math}
(See: Figure \ref{fig:meme2})
\begin{figure}[h]
    \centering
    \includegraphics[width=0.3\textwidth]{Pictures/meme.png} 
    \caption{Graph of this equation}
    \label{fig:meme2}
\end{figure}
\newpage
