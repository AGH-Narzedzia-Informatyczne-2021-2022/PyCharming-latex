\newpage
\section{\textsc{\textbf{\LARGE Our Website}}}
\label{sec:website}

\par We are working on our single page website, where you can find everything about our project and some instruction for our Discord Bot
The \underline{CharmingGuy} here is the permanent link \href{https://dhunanyan.github.io/PyCharming/}{PyCharming}

\subsection{What can you find on our \textbf{Website}}
\begin{enumerate}
    \item The Bot:
        \begin{itemize}
            \item Instruction
            \item API
            \item The actual code
        \end{itemize}
    \item Whole project:
        \begin{itemize}
            \item Our GitHub
            \item Our Discord for the Bot tests
            \item Our Overleaf project
            \item Our Google Docs
        \end{itemize}
\end{enumerate}

\subsection{Technologies that were used}
\begin{table}[h]
\centering
\begin{tabular}{|l|l|}
\cline{1-2}
& \\ \multicolumn{1}{|c|}{\textbf{Technologies}} & \multicolumn{1}{c|}{\textbf{Usage (in percents)}}\\ & \\ \hline \hline
HTML & 24.2\%  \\ \hline
SCSS & 47.6\% \\ \hline
JavaScript & 15.0\%  \\ \hline
CSS & 11.1\% \\ \hline
\end{tabular}
\label{tab:website}
\caption{Languages}
\end{table}


\subsubsection{What are programming languages?}

\par Programming languages are one kind of computer language, and are used in \underline{computer programming} to implement algorithms.
Most programming languages consist of instructions for computers. 
There are programmable machines that use a set of specific instructions, rather than general programming languages.

\newpage
\subsection{Some screenshots of our Website}
\begin{figure*}[ht!]
    \includegraphics[width=0.5\textwidth, height=90px]{Pictures/website1.png} \hfill
    \includegraphics[width=0.5\textwidth, height=90px]{Pictures/website3.png}
    \includegraphics[width=1\textwidth, height=180px]{Pictures/website2.png}
\end{figure*}

\subsection{Math Formula}
\par By the way, the only reason you see this formula here is to \underline{follow the requirment} and to \underline{to get the box ticked}.
\begin{align*}
S(\omega) 
&= \frac{\alpha g^2}{\omega^5} e^{[ -0.74\bigl\{\frac{\omega U_\omega 19.5}{g}\bigr\}^{\!-4}\,]} \\
&= \frac{\alpha g^2}{\omega^5} \exp\Bigl[ -0.74\Bigl\{\frac{\omega U_\omega 19.5}{g}\Bigr\}^{\!-4}\,\Bigr] 
\end{align*}